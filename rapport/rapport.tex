\documentclass [10pt,a4paper]{article}
\usepackage[danish]{babel}
\usepackage{a4wide}
\usepackage[T1]{fontenc}
\usepackage[utf8x]{inputenc}
\usepackage{amsmath}
\usepackage{amsfonts}
\usepackage{fancyhdr}
\usepackage{ucs}
\usepackage{graphicx}

\pagestyle{fancy}
\fancyhead[LO]{Daniel Egeberg \& Philip Munksgaard}
\fancyhead[RO]{OSM: G2}
\fancyfoot[CO]{\thepage}


\title{G2}
\author{Daniel Egeberg \& Philip Munksgaard}
\date{21. februar 2011}

\begin{document}
\maketitle

\section*{Opgave 1} % {{{

Til at repræsentere bruger-processer har vi valgt at lave \verb+struct+
indeholdende følgende informationer:

\begin{description}
    \item[state:] Processens tilstand. Kan være én af følgende:
        \begin{description}
            \item[FREE:] Processen er ikke tilknyttet noget program.
            \item[RUNNING:] Processen er tilknyttet et program, og kan køres
            \item[ZOMBIE:] Processen har kørt færdig, men dets returværdi er
                ikke blevet samlet op af et \verb+join+.
        \end{description}
    \item[name:] Processens filnavn.
    \item[return\_value:] Returværdien for processen.
\end{description}

Disse gemmes i en tabel hvor indekset er processens ID.

% }}}

\section*{Opgave 2} % {{{

% }}}

\end{document}
